         






\documentclass{article}

\begin{document}

\title{NATIONAL  IDENTIFICATION  NUMBER VERIFICATION}
\maketitle

\section {Executive summary }
{The problem this project will tackle is to save our citisens from the long queus and time they usually spend while waiting to be registered for the National Identification Number and in this,the person waiting has to be verified by the people in charge and this is susceptible to errors as data is entered into the system   .Therefor our citisens will only take a camera shot of their face and also insert their personal information.

Quick registration  is a great electronic data entry  for this, and is very free. In just three quick steps, one can capture a shot on their phone by simply holding their phone directly to their face using the front  camera and the act is known as the selfie or use a collegue to take your shot. Then select `load snap’. For best results, you have to face directly into the camera screen and Load. The registration concept  uses optical image  recognition, a fancy computer science process that involves the electronic conversion of photographed images into a computer readable image data .
Just like Quick registration process , we hope to address the same problem and to additionally give this project an ability that enables it to automatically activate the phone’s LED Flashlight/torch to ensure optimal lighting whenever picture is being taken, and to provide the additional functionality of allowing the user to feed in  his/her personal data information like surname,givenname and age e.t.c.
}







\section{ Aim and Objectives}

 \section{General objective.}
{To develop a mobile electronic data with an interface that will enable users to load the captured shot of their face together with their person information  without having to line in the long qeues as they wait for registration by NIN officers.
}


\section{ Specific Objectives}
{
  To analyze the user requirements of the existing system and identify the weaknesses thereof.
  To design an improved version of the mobile data entry interface.
  To implement and document the improved mobile data entry interface.
  To test and validate the user requirements of the improved electronic data interface.
}

 \section{Scope}
{The target group of this project is people with smart phones that possess cameras and LED flashlights.
}

 \section{Research Significance}
{This study is important because the added features of this interface will give the user an option to send his data plus his picture to the NIN information data head office and this will secure the person in the nation.
}

    
\section{ METHODOLOGY}

\section{Introduction.}
{This section gives an insight of how we plan on executing this report. We are going to look at the Research design, Source of data, Study sample,Data Collection tools, Systems Requirements Specification, Systems Design, Systems Testing and System implementation. After looking at this section, you will be able to draw conclusions about the feasibility of the application system project}.
\section{Research Design} 
{This constitutes the blue print of the collection, measurement, and analysis of data. We will employ the cross-sectional design method that involves collection of data and analysis of the collected data from a population or a representative subset, at one specific point in time. The data we collect will be cross-sectional data that we believe will represent the total population to a great extent.}
\section{Research Methods}
{As a group, we are using the qualitative method for carrying out our research, which involves understanding and explaining the social phenomena.}
\section{Sources of data}  
{Throughout this report we will concentrate on two major data sources, Secondary data and Primary data. 
The secondary data will be obtained from published journals, textbooks and other past students research into this field of interest. We will take keen interest in the documents relating to earlier developed application in the sector ranging from the most notable one, Quick registration.  
The Primary data will be got from questionnaires that will be handed out to the users that may hold valuable information on how to better our application. This will help us get straightforward and genuine answers to some of the queries we may have for them.

 The other way to engage the users will be through interviews. The interviews will help us get a broader understanding of the users’ experience with my; prototype. This way we can determine the level of satisfaction the users get from the application, how hard or easy some of the functions are. 
The other method we will use is observation, to get a feel of the user actually using a prototype of the application to load airtime.  Lastly, we will add some data from our own personal experience with the prototype.}
\section{Study sample} 
{The main data collection method we will employ in this study will be sampling. Sampling will help us to get data from a wide range of individuals that are involved directly or indirectly with the study. This will save a lot of time during data collection and will make the process of data analysis easy and less hectic. We will categories the participants according to age, sex and smart phone usage. From each category we will have a total of 10-20 samples giving us a total sample size of approximately 30-50 respondents. We believe this will give a proper representation of the group of users that we plan to develop the application for. 
To analyze the user requirements of the existing systems and identify there weaknesses, we use different methods to collect data, which include the following:}
\section{Data Collection methods} 
{We are going to employ more than two research methods namely; interviews that will require face to face interaction between us and the respondents and questionnaires. The questionnaires will be comprised of open ended and the close ended that will help us dictate the range of responses we need from the user. We will be applying each of these two methods interchangeably in order to get all the necessary data from our users.} 
\section{Interviews} 
{Here Face to face interactions will be held between we, the researchers and the respondents, call them users. We shall ask questions which are designed in the interview guide regarding our prototype. The respondents shall be expected to provide the necessary answers that we consider to be of value to us in the development process. We will use a pen and note book to record the information being given by the respondents to the various questions in the interview guide. Prior to conducting these interviews, we will use an interview schedule. We will be asking mainly the questions that were too broad to be included in the questionnaires. The interviews are going to provide instant answers to most of our queries that will better guide us in the overall development of the project.}
\section{Questionnaires} 
{Questionnaires are a list of predefined questions that have long been used by researchers to collect information from the users. It contains a list of predefined questions that the researcher is interested in. 
With this data collection method we will use the questionnaire guide to get a wide range of answers from a relatively large group of people that will be representing the whole population. This is a very simple and fast way of acquiring information from the population.  In our case, since our target group is people with smart phones that possess cameras and LED flashlights, we will find need to use both types of questionnaire approach. We are going to use both open-ended and closed-ended questionnaires. } 
\section{Open ended questionnaires.}
{With this type of questionnaire we will give the respondents complete control over what they would like to respond to and how they would like to respond to them. Here no restrictions are imposed on the respondents and we have no influence over their responses. The questions here will be totally related to loading airtime and other related airtime products.} 
\section{Close ended questionnaires.} 
{These allow us to get very specific responses because we will predefine a list of expected responses that a respondent may be able to choose from. Here we greatly influence the user’s responses by giving him/her a list of acceptable responses. Some responses include yes or no, true or false answers. These will enable us to easily analyze the data collected. From this questionnaire will be concentrating on the very important parts of the User Interface Design (UID) and level of satisfaction from the prototype.}
\section{Observation}
{We are also going to employ observation of users using the application as a method of data collection. Observation will clearly tell us what we need to know about the user’s reaction as he/she loads airtime with our application, and the amount of time taken while doing so. Here, we will mainly be interested in the user experience in comparison to existing applications in the line of our study. From the data collected, we will be in position to better the user experience for our application that suits their needs. We will also be able to know the key areas that interest the user. The goal here is so that we develop something that is easy to use and time efficient at the same time.
There are different types of observation research approaches but as a group, we shall mainly use two techniques, that is to say:
 \section{Overt observational research.} 
{This is a type of observational research whereby the researchers identify themselves as researchers and explain the purpose of their observations. The problem with this approach is subjects may modify their behavior when they know they are being watched. They portray their “ideal self” rather than their true self. The advantage that the overt approach has over the covert approach is that there is no deception.
As a group we plan to go out to the public aiming for our target group of smart phone users and present ourselves as researchers providing them with a prototype of the application that they will use as we observe them. Through this, we will be able to gather the data we need to improve our application.}




\section{ Researcher Participation.} 
In this type of observation approach the researcher participates in what they are observing so as to get a finer appreciation of the phenomena. As a group we will each take turns at using the application, so that as one person loads airtime using the application, the others will observe. At the end of it all, each and every person will have firsthand experience of using the application and also everyone will\section have data gathered from observing. We shall compile the different data, and with this we hope to find out what we can do as developers to better the application.




{Data analysis}
In addition to the notes made after each interview, each transcript will be reviewed many times and themes, patterns and insight will be documented. When this process is completed, similar ideas and themes will be grouped and given a conceptual label


\section{SYSTEM DESIGN}

{To design an improved version of the mobile application we will use process modeling system design where we will use fact-finding techniques such as interviews and questionnaires to investigate the current system and identify user requirements.}

\section{Process modeling}
{Process modeling is a technique for organizing and documenting the structure and flow of dta through a systems processes and/ the logic, policies, and procedures to be implemented by a system’s processes.} 
\section{A system analysis process model consists of data flow diagrams.}
{
A data flow diagram (DFD) is a tool that depicts the flow of data through a system and the work or processing performed by that system[   ].
 Process modeling involves three tools but as a group we will use data flow diagrams which show how data moves through the system but does not show the program logic.}

\section{Data modeling} 
{A data model is a description of how data should be used to meet the requirements given by the en d users. It helps to understand the system requirements and explore data oriented structures.
In 1970 Peter Chen invented and introduced the entity relationship modeling technique which we shall use to achieve data modeling.
}

\section{System implementation}
{Systems implementation is the construction of the new system and the delivery of that system into production [   ]
}
\section{System implementation tools}
{In the implementation of our system, we will use the following tools:-}
  \section    {Interfaces}
{The interfaces will be implemented using Java and XML programming languages. These interfaces include: Load Airtime For Self/Load For Another, Edit Digits, Subscribe For SMSs, Subscribe For Data Bundles. 
This our system will run on an android platform}

\section{System Testing and Validation}
{The system will be checked for errors by running it using program emulators. 
The system will also be validated by allowing time for users to interact with the application under observation.
}
\section{Conclusion}
{The above methods, from the research design to system testing and validation, will lead to the realization of the quick national identification number registration.}
 

\section{ References}

. Document scanner retrieved 16thApril, 2016 from,
https://play.google.com/store/apps/details?id=com.senso.documentscanner&hl=en.       

.  National identification and Registration Authority
 At Kololo airstrip.




\end{document}


